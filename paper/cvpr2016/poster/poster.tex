\documentclass[landscape,final,a0paper,fontscale=0.285]{baposter}

\usepackage{calc}
\usepackage{graphicx}
\usepackage{amsmath}
\usepackage{amssymb}
\usepackage{relsize}
\usepackage{multirow}
\usepackage{rotating}
\usepackage{bm}
\usepackage{url}


\usepackage{graphicx}
\usepackage{multicol}

\usepackage{gensymb}

\usepackage{bbm}

\usepackage{booktabs}

%\usepackage{times}
%\usepackage{helvet}
%\usepackage{bookman}
\usepackage{palatino}

\newcommand{\captionfont}{\footnotesize}

\graphicspath{{images/}{../images/}}
\usetikzlibrary{calc}

\newcommand{\SET}[1]  {\ensuremath{\mathcal{#1}}}
\newcommand{\MAT}[1]  {\ensuremath{\boldsymbol{#1}}}
\newcommand{\VEC}[1]  {\ensuremath{\boldsymbol{#1}}}
\newcommand{\Video}{\SET{V}}
\newcommand{\video}{\VEC{f}}
\newcommand{\track}{x}
\newcommand{\Track}{\SET T}
\newcommand{\LMs}{\SET L}
\newcommand{\lm}{l}
\newcommand{\PosE}{\SET P}
\newcommand{\posE}{\VEC p}
\newcommand{\negE}{\VEC n}
\newcommand{\NegE}{\SET N}
\newcommand{\Occluded}{\SET O}
\newcommand{\occluded}{o}

\newcommand{\by}[2]{\ensuremath{#1 \! \times \! #2}}

\newcommand{\thetav}{\ensuremath{\bm{\theta}}}
\newcommand{\dv}{\ensuremath{\bm{d}}}
\newcommand{\gv}{\ensuremath{\bm{g}}}
\newcommand{\xv}{\ensuremath{\bm{x}}}
\newcommand{\yv}{\ensuremath{\bm{y}}}
\newcommand{\zv}{\ensuremath{\bm{z}}}



\DeclareRobustCommand\onedot{\futurelet\@let@token\@onedot}


%%%%%%%%%%%%%%%%%%%%%%%%%%%%%%%%%%%%%%%%%%%%%%%%%%%%%%%%%%%%%%%%%%%%%%%%%%%%%%%%
%%%% Some math symbols used in the text
%%%%%%%%%%%%%%%%%%%%%%%%%%%%%%%%%%%%%%%%%%%%%%%%%%%%%%%%%%%%%%%%%%%%%%%%%%%%%%%%

%%%%%%%%%%%%%%%%%%%%%%%%%%%%%%%%%%%%%%%%%%%%%%%%%%%%%%%%%%%%%%%%%%%%%%%%%%%%%%%%
% Multicol Settings
%%%%%%%%%%%%%%%%%%%%%%%%%%%%%%%%%%%%%%%%%%%%%%%%%%%%%%%%%%%%%%%%%%%%%%%%%%%%%%%%
%\setlength{\columnsep}{1.5em}
%\setlength{\columnseprule}{0mm}

%%%%%%%%%%%%%%%%%%%%%%%%%%%%%%%%%%%%%%%%%%%%%%%%%%%%%%%%%%%%%%%%%%%%%%%%%%%%%%%%
% Save space in lists. Use this after the opening of the list
%%%%%%%%%%%%%%%%%%%%%%%%%%%%%%%%%%%%%%%%%%%%%%%%%%%%%%%%%%%%%%%%%%%%%%%%%%%%%%%%
\newcommand{\compresslist}{%
\setlength{\itemsep}{1pt}%
\setlength{\parskip}{0pt}%
\setlength{\parsep}{0pt}%
}

%%%%%%%%%%%%%%%%%%%%%%%%%%%%%%%%%%%%%%%%%%%%%%%%%%%%%%%%%%%%%%%%%%%%%%%%%%%%%%
%%% Begin of Document
%%%%%%%%%%%%%%%%%%%%%%%%%%%%%%%%%%%%%%%%%%%%%%%%%%%%%%%%%%%%%%%%%%%%%%%%%%%%%%

\begin{document}

%%%%%%%%%%%%%%%%%%%%%%%%%%%%%%%%%%%%%%%%%%%%%%%%%%%%%%%%%%%%%%%%%%%%%%%%%%%%%%
%%% Here starts the poster
%%%---------------------------------------------------------------------------
%%% Format it to your taste with the options
%%%%%%%%%%%%%%%%%%%%%%%%%%%%%%%%%%%%%%%%%%%%%%%%%%%%%%%%%%%%%%%%%%%%%%%%%%%%%%
% Define some colors

%\definecolor{lightblue}{cmyk}{0.83,0.24,0,0.12}

\definecolor{lightblue}{rgb}{0.145,0.6666,1}
\definecolor{blue}{RGB}{150,200,242}
\definecolor{darkblue}{RGB}{68,92,170}
%\definecolor{brown}{RGB}{245,236,215}
%\definecolor{white}{RGB}{255,255,255}

%\definecolor{lightblue}{rgb}{0.5,1,0.25}
%\definecolor{lightblue}{rgb}{1,0.2,0.4}
%\definecolor{lightblue}{rgb}{0,0.85,1}

% Draw a video
%\newlength{\FSZ}
%\newcommand{\drawvideo}[3]{% [0 0.25 0.5 0.75 1 1.25 1.5]
%   \noindent\pgfmathsetlength{\FSZ}{\linewidth/#2}
%   \begin{tikzpicture}[outer sep=0pt,inner sep=0pt,x=\FSZ,y=\FSZ]
%   \draw[color=lightblue!50!black] (0,0) node[outer sep=0pt,inner sep=0pt,text width=\linewidth,minimum height=0] (video) {\noindent#3};
%   \path [fill=lightblue!50!black,line width=0pt] 
%     (video.north west) rectangle ([yshift=\FSZ] video.north east) 
%    \foreach \x in {1,2,...,#2} {
%      {[rounded corners=0.6] ($(video.north west)+(-0.7,0.8)+(\x,0)$) rectangle +(0.4,-0.6)}
%    }
%;
%   \path [fill=lightblue!50!black,line width=0pt] 
%     ([yshift=-1\FSZ] video.south west) rectangle (video.south east) 
%    \foreach \x in {1,2,...,#2} {
%      {[rounded corners=0.6] ($(video.south west)+(-0.7,-0.2)+(\x,0)$) rectangle +(0.4,-0.6)}
%    }
%;
%   \foreach \x in {1,...,#1} {
%     \draw[color=lightblue!50!black] ([xshift=\x\linewidth/#1] video.north west) -- ([xshift=\x\linewidth/#1] video.south west);
%   }
%   \foreach \x in {0,#1} {
%     \draw[color=lightblue!50!black] ([xshift=\x\linewidth/#1,yshift=1\FSZ] video.north west) -- ([xshift=\x\linewidth/#1,yshift=-1\FSZ] video.south west);
%   }
%   \end{tikzpicture}
%}

\hyphenation{resolution occlusions}
%%
\begin{poster}%
  % Poster Options
  {
  % Show grid to help with alignment
  grid=false,
  % Column spacing
  columns=3,
  colspacing=1em,
  % Color style
  headerColorOne=blue,
  borderColor=darkblue,
  bgColorOne=white,
  boxColorOne=white,
  % Format of textbox
  textborder=roundedleft,
  % Format of text header
  eyecatcher=true,
  headerborder=open, %closed,
  headerheight=0.12\textheight,
%  textfont=\sc, An example of changing the text font
  headershape=roundedright,
  headershade=plain, %shadelr,
  headerfont=\Large\bf\textsc, %Sans Serif
  textfont={\setlength{\parindent}{1.5em}},
  boxshade=plain,
%  background=shade-tb,
  background=plain,
  linewidth=2pt
  }
  % Eye Catcher
  %{\includegraphics[height=5em]{images/graph_occluded.pdf}} 
  {}
%  \includegraphics[height=6em]{fig/logo/ucla2.png} }
%   \includegraphics[height=6em]{fig/logo/ttic.png}}
  % Title
  { {\huge Attention to Scale: Scale-aware Semantic Image Segmentation} }
  % Authors
  %{{\{ Brian.Amberg and Thomas.Vetter \}@unibas.ch}}
  { \vspace{0.3cm} {\Large
      \begin{tabular}{ccccc} 
        Liang-Chieh Chen\textsuperscript{1} & Yi Yang\textsuperscript{2} & Jiang Wang\textsuperscript{2} & Wei Xu\textsuperscript{2} & Alan L. Yuille\textsuperscript{1,3} \\
      & \textsuperscript{1}UCLA & \textsuperscript{2}Baidu USA & \textsuperscript{3} JHU
    \end{tabular} }
  }
  % University logo
  {}%   \includegraphics[height=4em]{fig/logo/google.pdf} }
%  {\includegraphics[height=6em]{fig/logo/ut.png}}
  %{% The makebox allows the title to flow into the logo, this is a hack because of the L shaped logo.
  %  \includegraphics[height=9.0em]{images/logo}
  %}

%%%%%%%%%%%%%%%%%%%%%%%%%%%%%%%%%%%%%%%%%%%%%%%%%%%%%%%%%%%%%%%%%%%%%%%%%%%%%%
%%% Now define the boxes that make up the poster
%%%---------------------------------------------------------------------------
%%% Each box has a name and can be placed absolutely or relatively.
%%% The only inconvenience is that you can only specify a relative position 
%%% towards an already declared box. So if you have a box attached to the 
%%% bottom, one to the top and a third one which should be in between, you 
%%% have to specify the top and bottom boxes before you specify the middle 
%%% box.
%%%%%%%%%%%%%%%%%%%%%%%%%%%%%%%%%%%%%%%%%%%%%%%%%%%%%%%%%%%%%%%%%%%%%%%%%%%%%%
    %
    % A coloured circle useful as a bullet with an adjustably strong filling
  \newcommand{\colouredcircle}{%
    \tikz{\useasboundingbox (-0.2em,-0.32em) rectangle(0.2em,0.32em); \draw[draw=black,fill=lightblue,line width=0.03em] (0,0) circle(0.18em);}}

%%%%%%%%%%%%%%%%%%%%%%%%%%%%%%%%%%%%%%%%%%%%%%%%%%%%%%%%%%%%%%%%%%%%%%%%%%%%%%
  \headerbox{Motivation}{name=intro,column=0}{
%%%%%%%%%%%%%%%%%%%%%%%%%%%%%%%%%%%%%%%%%%%%%%%%%%%%%%%%%%%%%%%%%%%%%%%%%%%%%%
    Multi-scale features key of s-o-a semantic segmentation models.

    \centering
    \begin{tabular}{c c c}
      {
        \includegraphics[width=0.2\linewidth]{fig/attention/voc12/img/2008_008221.jpg}} &
      {
        \includegraphics[width=0.2\linewidth]{fig/attention/voc12/img/2007_002728.jpg}} &      
      {
        \includegraphics[width=0.15\linewidth]{fig/attention/coco/img/COCO_val2014_000000000395.jpg}} \\
      {            
        \includegraphics[width=0.2\linewidth]{fig/attention/voc12/gt/2008_008221.png}} &
      {
        \includegraphics[width=0.2\linewidth]{fig/attention/voc12/gt/2007_002728.png}} &
      {
        \includegraphics[width=0.15\linewidth]{fig/attention/coco/gt_tmp/COCO_val2014_000000000395.png}} \\
      {\small
      (a) small-scale person} &
      {\small
        (b) large-scale person} &
      {\small
        (c) persons of} \\
      {\small
        large-scale train} &
      {\small
        small-scale train} &
      {\small
        several scales} \\
    \end{tabular}
%  }

}

%%%%%%%%%%%%%%%%%%%%%%%%%%%%%%%%%%%%%%%%%%%%%%%%%%%%%%%%%%%%%%%%%%%%%%%%%%%%%%
  \headerbox{Model illustration}{name=modelIllustration,column=0,below=intro}{
%%%%%%%%%%%%%%%%%%%%%%%%%%%%%%%%%%%%%%%%%%%%%%%%%%%%%%%%%%%%%%%%%%%%%%%%%%%%%%
    {
    \centering
    \begin{tabular}{c}
      \includegraphics[width=0.85\linewidth]{fig/attention/model_illustration/fig4.pdf} \\
    \end{tabular}

    \centering
    \begin{tabular}{c}
      \includegraphics[width=0.85\linewidth]{fig/attention/model_illustration/fig5_2.pdf} \\
    \end{tabular}
    
    \centering
    \begin{tabular}{c}
      \includegraphics[width=0.9\linewidth]{fig/attention/model_illustration/our_fig4.pdf} \\
    \end{tabular}
    }

  }
 
%%%%%%%%%%%%%%%%%%%%%%%%%%%%%%%%%%%%%%%%%%%%%%%%%%%%%%%%%%%%%%%%%%%%%%%%%%%%%%
\headerbox{Attention model: multi-scale features}{name=att,column=1}{
%%%%%%%%%%%%%%%%%%%%%%%%%%%%%%%%%%%%%%%%%%%%%%%%%%%%%%%%%%%%%%%%%%%%%%%%%%%%%%
  \begin{itemize}
        \setlength{\itemsep}{-5pt}
    \item Suppose input image resized to several scales $s \in \{1, ..., S\}$. 
    \item Input with scale $s$ produces a score map $f^s_{i,c}$, ($i$ over pixels, and $c$ over object classes).
    \item Let $g_{i,c}$ be the weighted sum of score maps at $(i,c)$ for all scales
  \end{itemize}

    \begin{equation}
      \label{eq:weighted_sum}
      g_{i,c} = \sum_{s=1}^S w^s_{i} \cdot f^s_{i,c}
    \end{equation}
    
    The weight $w^s_i$ is computed by
    \begin{equation}
      w^s_{i} = \frac{\exp(h^s_{i})}{\sum_{t=1}^S \exp(h^t_{i})}
    \end{equation}
    where $h^s_{i}$ is score map by {\it attention} model.

    \vspace{-0.3cm}
    \begin{itemize}
              \setlength{\itemsep}{-5pt}
      \item $w^s_{i}$ reflects importance of feature at position $i$ and scale $s$.
      \item Visualize attention for each scale by visualizing $w^s_{i}$.
      \item Average- or max-pooling over scales are two special cases.
    \end{itemize}

    
}

%%%%%%%%%%%%%%%%%%%%%%%%%%%%%%%%%%%%%%%%%%%%%%%%%%%%%%%%%%%%%%%%%%%%%%%%%%%%%%
\headerbox{Learned attention: Max vs. Attention}{name=learned_att,column=1,row=0, below=att}{
%%%%%%%%%%%%%%%%%%%%%%%%%%%%%%%%%%%%%%%%%%%%%%%%%%%%%%%%%%%%%%%%%%%%%%%%%%%%%%
    \centering          
    \scalebox{1}{
    \begin{tabular}{c c c | c c c}
      & \includegraphics[height=0.1\linewidth, width=0.1\linewidth]{fig/attention/voc10_part/img/2008_000034.jpg} & & & \includegraphics[height=0.1\linewidth, width=0.1\linewidth]{fig/attention/voc10_part/img/2008_003344.jpg} & \\
      {\includegraphics[height=0.1\linewidth, width=0.12\linewidth]{fig/attention/voc10_part/att1/2008_000034_max.pdf}} &
      {\includegraphics[height=0.1\linewidth, width=0.12\linewidth]{fig/attention/voc10_part/att2/2008_000034_max.pdf}} &
      {\includegraphics[height=0.1\linewidth, width=0.12\linewidth]{fig/attention/voc10_part/att3/2008_000034_max.pdf}} &
      {\includegraphics[height=0.1\linewidth, width=0.12\linewidth]{fig/attention/voc10_part/att1/2008_003344_max.pdf}} &
      {\includegraphics[height=0.1\linewidth, width=0.12\linewidth]{fig/attention/voc10_part/att2/2008_003344_max.pdf}} &
      {\includegraphics[height=0.1\linewidth, width=0.12\linewidth]{fig/attention/voc10_part/att3/2008_003344_max.pdf}} \\
      {\includegraphics[height=0.1\linewidth, width=0.12\linewidth]{fig/attention/voc10_part/att1/2008_000034.pdf}} &
      {\includegraphics[height=0.1\linewidth, width=0.12\linewidth]{fig/attention/voc10_part/att2/2008_000034.pdf}} &
      {\includegraphics[height=0.1\linewidth, width=0.12\linewidth]{fig/attention/voc10_part/att3/2008_000034.pdf}} &
      {\includegraphics[height=0.1\linewidth, width=0.12\linewidth]{fig/attention/voc10_part/att1/2008_003344.pdf}} &
      {\includegraphics[height=0.1\linewidth, width=0.12\linewidth]{fig/attention/voc10_part/att2/2008_003344.pdf}} &
      {\includegraphics[height=0.1\linewidth, width=0.12\linewidth]{fig/attention/voc10_part/att3/2008_003344.pdf}} \\
                      {{\tiny (a) Scale-1 Att.}} &
                      {{\tiny (b) Scale-0.75 Att.}} &
                      {{\tiny (c) Scale-0.5 Att.}} &
                      {{\tiny (a) Scale-1 Att.}} &
                      {{\tiny (b) Scale-0.75 Att.}} &
                      {{\tiny (c) Scale-0.5 Att.}} \\
    \end{tabular}
    }

    \begin{itemize}
      \setlength{\itemsep}{-5pt}
    \item {Scale-1 attention $\rightarrow$ small-scale objects.}
    \item {Scale-0.75 attention $\rightarrow$ middle-scale objects.}
    \item {Scale-0.5 attention $\rightarrow$ large-scale objects or background.}
    \end{itemize}
}


%%%%%%%%%%%%%%%%%%%%%%%%%%%%%%%%%%%%%%%%%%%%%%%%%%%%%%%%%%%%%%%%%%%%%%%%%%%%%%
\headerbox{Extra Supervision}{name=extra,column=1,row=0, below=learned_att}{
%%%%%%%%%%%%%%%%%%%%%%%%%%%%%%%%%%%%%%%%%%%%%%%%%%%%%%%%%%%%%%%%%%%%%%%%%%%%%%
  {
  \centering
  \begin{tabular}{c}
    \includegraphics[width=0.85\linewidth]{fig/attention/model_illustration/add_sup_fig4.pdf} \\
  \end{tabular}
  }

}

%%%%%%%%%%%%%%%%%%%%%%%%%%%%%%%%%%%%%%%%%%%%%%%%%%%%%%%%%%%%%%%%%%%%%%%%%%%%%%
\headerbox{PASCAL VOC 2012}{name=voc_test,column=2,row=0}{
%%%%%%%%%%%%%%%%%%%%%%%%%%%%%%%%%%%%%%%%%%%%%%%%%%%%%%%%%%%%%%%%%%%%%%%%%%%%%%
  \begin{tabular}{c}
    {
    \centering
    \addtolength{\tabcolsep}{2.5pt}
    \begin{tabular}{l c c}
      \toprule[0.2 em]
      \multicolumn{2}{l}{Baseline: DeepLab-LargeFOV} & 67.58  \\
      \toprule[0.2 em]
      {\bf Merging Method} &  & w/ E-Supv \\
      \midrule
      {\it Scales = \{1, 0.75, 0.5\}} & & \\
      Max-Pooling & 69.70 & 70.06 \\
      Average-Pooling & 68.82 & 70.55 \\
      Attention & 69.47 & {\bf 71.42} \\
      \bottomrule[0.1 em]
    \end{tabular}
    } \\
    (a) val set \\
    {
      \centering
      \addtolength{\tabcolsep}{2.5pt}
      \begin{tabular} {l | c}
        \toprule[0.2 em]
                {\bf Method} & mIOU \\
                \midrule \midrule
                DeepLab-CRF-COCO-LargeFOV & 72.7 \\
                DeepLab-MSc-CRF-COCO-LargeFOV & 73.6 \\
                \midrule
                DeepLab-CRF-COCO-LargeFOV-{\bf Attention} & 75.1 \\
                DeepLab-CRF-COCO-LargeFOV-{\bf Attention}+ & 75.7 \\
                \bottomrule[0.1 em]
      \end{tabular}
    } \\
    (b) test set \\
  \end{tabular}
}

%%%%%%%%%%%%%%%%%%%%%%%%%%%%%%%%%%%%%%%%%%%%%%%%%%%%%%%%%%%%%%%%%%%%%%%%%%%%%%
\headerbox{Segmentation Results}{name=seg_results,column=2,row=0, below=voc_test}{
%%%%%%%%%%%%%%%%%%%%%%%%%%%%%%%%%%%%%%%%%%%%%%%%%%%%%%%%%%%%%%%%%%%%%%%%%%%%%%
  \hspace{-0.8cm}
  \centering
  \scalebox{1}{
  \begin{tabular}{c c c c c c}
   \includegraphics[height=0.09\linewidth]{fig/attention/voc12/img/2010_004795.jpg} &
   {\includegraphics[height=0.09\linewidth]{fig/attention/voc12/res_baseline/2010_004795.png}} &
   {\includegraphics[height=0.09\linewidth]{fig/attention/voc12/res_sharenet/2010_004795.png}} &
   {\includegraphics[height=0.09\linewidth]{fig/attention/voc12/att1/2010_004795.pdf}} &
   {\includegraphics[height=0.09\linewidth]{fig/attention/voc12/att2/2010_004795.pdf}} &
   {\includegraphics[height=0.09\linewidth]{fig/attention/voc12/att3/2010_004795.pdf}} \\
   \includegraphics[height=0.09\linewidth]{fig/attention/voc12/img/2011_001642.jpg} &
      {\includegraphics[height=0.09\linewidth]{fig/attention/voc12/res_baseline/2011_001642.png}} &
      {\includegraphics[height=0.09\linewidth]{fig/attention/voc12/res_sharenet/2011_001642.png}} &
      {\includegraphics[height=0.09\linewidth]{fig/attention/voc12/att1/2011_001642.pdf}} &
      {\includegraphics[height=0.09\linewidth]{fig/attention/voc12/att2/2011_001642.pdf}} &
      {\includegraphics[height=0.09\linewidth]{fig/attention/voc12/att3/2011_001642.pdf}} \\
   \includegraphics[height=0.09\linewidth]{fig/attention/voc12/img/2011_002121.jpg} &
      {\includegraphics[height=0.09\linewidth]{fig/attention/voc12/res_baseline/2011_002121.png}} &
      {\includegraphics[height=0.09\linewidth]{fig/attention/voc12/res_sharenet/2011_002121.png}} &
      {\includegraphics[height=0.09\linewidth]{fig/attention/voc12/att1/2011_002121.pdf}} &
      {\includegraphics[height=0.09\linewidth]{fig/attention/voc12/att2/2011_002121.pdf}} &
      {\includegraphics[height=0.09\linewidth]{fig/attention/voc12/att3/2011_002121.pdf}} \\
   \includegraphics[height=0.09\linewidth]{fig/attention/voc12/img/2011_003256.jpg} &
      {\includegraphics[height=0.09\linewidth]{fig/attention/voc12/res_baseline/2011_003256.png}} &
      {\includegraphics[height=0.09\linewidth]{fig/attention/voc12/res_sharenet/2011_003256.png}} &
      {\includegraphics[height=0.09\linewidth]{fig/attention/voc12/att1/2011_003256.pdf}} &
      {\includegraphics[height=0.09\linewidth]{fig/attention/voc12/att2/2011_003256.pdf}} &
      {\includegraphics[height=0.09\linewidth]{fig/attention/voc12/att3/2011_003256.pdf}} \\
   \includegraphics[height=0.09\linewidth]{fig/attention/voc12/img/2007_008260.jpg} &
      {\includegraphics[height=0.09\linewidth]{fig/attention/voc12/res_baseline/2007_008260.png}} &
      {\includegraphics[height=0.09\linewidth]{fig/attention/voc12/res_sharenet/2007_008260.png}} &
      {\includegraphics[height=0.09\linewidth]{fig/attention/voc12/att1/2007_008260.pdf}} &
      {\includegraphics[height=0.09\linewidth]{fig/attention/voc12/att2/2007_008260.pdf}} &
      {\includegraphics[height=0.09\linewidth]{fig/attention/voc12/att3/2007_008260.pdf}} \\
      {\tiny (a) Image} & 
      {{\tiny (b) Baseline}} & 
      {{\tiny (c) Our model}} & 
      {{\tiny (d) Scale-1 Att.}} & 
      {{\tiny (e) Scale-0.75 Att.}} &
      {{\tiny (f) Scale-0.5 Att.}} \\
  \end{tabular}
  }
}


%%%%%%%%%%%%%%%%%%%%%%%%%%%%%%%%%%%%%%%%%%%%%%%%%%%%%%%%%%%%%%%%%%%%%%%%%%%%%%
\headerbox{Conclusion}{name=conc,column=2,row=0, below=seg_results}{
%%%%%%%%%%%%%%%%%%%%%%%%%%%%%%%%%%%%%%%%%%%%%%%%%%%%%%%%%%%%%%%%%%%%%%%%%%%%%%
\hspace{-8mm}\begin{minipage}{\linewidth}
  \begin{itemize}
    \setlength{\itemsep}{-5pt}
  \item Using multi-scale inputs > single scale input. 
  \item Attention model brings better performance and allows to visualize the importance of features.
  \item Adding extra supervision is essential for better performance.
  \item Try it out! Source code and trained models available at {\color{red} \url{http://liangchiehchen.com/projects/DeepLab.html}}.
  \end{itemize}
\end{minipage}
}
%%%%%%%%%%%%%%%%%%%%%%%%%%%%%%%%%%%%%%%%%%%%%%%%%%%%%%%%%%%%%%%%%%%%%%%%%%%%%%
\headerbox{References}{name=references,column=2,below=conc}{
%%%%%%%%%%%%%%%%%%%%%%%%%%%%%%%%%%%%%%%%%%%%%%%%%%%%%%%%%%%%%%%%%%%%%%%%%%%%%%
   \vspace{-0.1cm}
    \tiny
    %\scriptsize
    \bibliographystyle{ieee}
    \renewcommand{\section}[2]{\vskip -0.05em}
      \begin{thebibliography}{1}\itemsep=-0.7em
      \setlength{\baselineskip}{0.3em}
      \bibitem{farabet2013learning}
        C.~Farabet et~al.
        \newblock Learning hierarchical features for scene labeling.
        \newblock {\em PAMI}, 2013.
      \bibitem{bahdanau2014neural}
        D.~Bahdanau, K.~Cho, and Y.~Bengio.
        \newblock Neural machine translation by jointly learning to align and
        translate.
        \newblock In {\em ICLR}, 2015.
      \bibitem{lin2015efficient}
        G.~Lin et~al.
        \newblock Efficient piecewise training of deep structured models for semantic
        segmentation.
        \newblock {\em arXiv:1504.01013}, 2015.

      \end{thebibliography}
}


\end{poster}

\end{document}
